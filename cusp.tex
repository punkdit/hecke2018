
\documentclass[11pt,oneside]{article} 

\usepackage{a4wide}

\usepackage{amsmath}
\usepackage{color}
%\usepackage{natbib} % kills arxiv 
\usepackage{framed}
%\usepackage{cite}
\usepackage{tikz}
\usepackage{tikz-cd}

% http://www.maths.qmul.ac.uk/~mf/genyoungtabtikz.sty
\usepackage[vcentermath]{genyoungtabtikz}
\Yboxdim{8pt}
%\Ylinethick{1pt}

\RequirePackage{amsmath}
\RequirePackage{amssymb}
\RequirePackage{amsthm}

%\RequirePackage{algorithmic}
%\RequirePackage{algorithm}
%\RequirePackage{theorem}
%\RequirePackage{eucal}
\RequirePackage{color}
\RequirePackage{url}
\RequirePackage{mdwlist}

\RequirePackage{rotating}


\RequirePackage[all]{xy}
%\_CompileMatrices
%\RequirePackage{hyperref}
\RequirePackage{graphicx}
%\RequirePackage[dvips]{geometry}

\usepackage{xcolor}
%\usepackage{amsmath,amsfonts,amssymb}
\usepackage{graphicx}
\usepackage[caption=false]{subfig}
\usepackage{enumerate}
\usepackage{mathrsfs}
\usepackage{bbm}

% -------------- Commands ----------------------

\makeatletter
\newcommand{\verbatimfont}[1]{\renewcommand{\verbatim@font}{\ttfamily#1}}
\makeatother

\newcommand{\Eref}[1]{(\ref{#1})}
\newcommand{\Fref}[1]{Fig.~\ref{#1}}
%\newcommand{\Aref}[1]{Appendix~\ref{#1}}
\newcommand{\SRef}[1]{Section~\ref{#1}}

\newcommand{\todo}[1]{\ \textcolor{red}{\{#1\}}\ }

\newcommand{\Aut}{\mathrm{Aut}}
\newcommand{\Hom}{\mathrm{Hom}}
%\newcommand{\hom}{\mathrm{hom}} % internal hom ?
\newcommand{\Stab}{\mathrm{Stab}}
\newcommand{\Fix}{\mathrm{Fix}}
\newcommand{\Orbit}{\mathrm{Orbit}}
\newcommand{\Ker}{\mathrm{Ker}}
\newcommand{\Image}{\mathrm{Im}}
\newcommand{\Dim}{\mathrm{Dim}}
\newcommand{\Complex}{\mathbb{C}}
\newcommand{\Integer}{\mathbb{Z}}
\newcommand{\Natural}{\mathbb{N}}

\newcommand{\GL}{\mathrm{GL}}
\newcommand{\SL}{\mathrm{SL}}
\newcommand{\SO}{\mathrm{SO}}
\newcommand{\Sp}{\mathrm{Sp}}
\newcommand{\PGL}{\mathrm{PGL}}
\newcommand{\Field}{\mathbb{F}}

% Lemma, proof, theorem, etc.
\newcommand\nounderline[1]{ #1} 
\newcommand\dodef[1]{\vskip 5pt \noindent{\bf \underline{Definition #1.}\ }}
\newcommand\dolemma[1]{\vskip 5pt \noindent{\bf \underline{Lemma #1.}\ }}
\newcommand\doproposition[1]{\vskip 5pt \noindent {\bf \underline{Proposition #1.}\ }}
\newcommand\dotheorem[1]{\vskip 5pt \noindent {\bf \underline{Theorem #1.}\ }}
\newcommand\docorollary[1]{\vskip 5pt \noindent {\bf \underline{Corollary #1.}\ }}
\newcommand\doexample[1]{\vskip 5pt \noindent {\bf \underline{Example #1.}\ }}
\newcommand\doproof{\vskip 5pt \noindent{\bf \nounderline{Proof:}\ }}

\newcommand\tombstone{\rule{.36em}{2ex}\vskip 5pt}

\newcounter{numitem}
\newcommand{\numlabel}[1]{\refstepcounter{numitem}\thenumitem\label{#1}}
\newcommand{\numitem}{\refstepcounter{numitem}\thenumitem}

% Categories
\newcommand{\Set}{\mathbf{Set}}
\newcommand{\FinSet}{\mathbf{FinSet}}
\newcommand{\GSet}{\mathbf{GSet}}
\newcommand{\GRep}{\mathbf{GRep}}
\newcommand{\CRing}{\mathbf{CRing}}

\newcommand{\thinplus}{\!+\!}

\newcommand{\tensor}{\otimes}

%\input{dynkin.tex}


%\title{The philosophy of cusp forms?}
\title{Parabolic Induction}

%\author{Sijato Budotr}
\author{Simon Burton}

\date{\today}

\flushbottom

\begin{document}

\maketitle

\section{$\GL(n,\Field_p)$}

We are trying to make sense of chapter 47 in \cite{Bump2004}.
Part of this is the correspondence (duality)
between the irreducible complex
representations of a group $G$ and the conjugacy classes of the dual group $G^{\vee}$.
And also \cite{Joyal1995}.

Elements of a 
conjugacy class of $\GL(n,\Field_p)$ have the same characteristic polynomial.


%$$
%\begin{array}{cc}
%\end{array}
%$$

\setlength{\tabcolsep}{10pt}
\setlength{\arraycolsep}{1pt}
\renewcommand{\arraystretch}{0.5}

%\begin{center}
%\begin{tabular}{c|c}
%\# & char. poly \\
%\hline
%1 & $(x+1)^3$ \\
%21 & $(x+1)^3$ \\
%24 & $x^3+x^2+1$ \\
%24 & $x^3+x+1$ \\
%42 & $(x+1)^3$ \\
%56 & $(x+1)(x^2+x+1)$ \\
%\end{tabular}
%\end{center}

$G=\GL(1,\Field_2)$ has order 1, with 1 conjugacy class:
\begin{center}
\begin{tabular}{r|l|c|c|c}
\# & representative & char. polynomial & irrep & dim. \\
\hline
1 & $[1]$           & $x+1$            & $A$   & 1    \\
\end{tabular}
\end{center}

$G=\GL(2,\Field_2)$ has order 6, with 3 conjugacy classes:
\begin{center}
\begin{tabular}{r|l|c|c|c}
\# & representative(s) & char. polynomial & irrep & dim. \\
\hline
1  & $\begin{bmatrix}1&.\\.&1\end{bmatrix}$    & $(x+1)^2$  & $\yng(1,1)(A)$  & 1 \\
2  & $\begin{bmatrix}1&1\\1&.\end{bmatrix}$, $\begin{bmatrix}.&1\\1&1\end{bmatrix}$     & $x^2+x+1$  &  $B$  & 1 \\
3  & $\begin{bmatrix}1&1\\.&1\end{bmatrix}$    & $(x+1)^2$  & $\yng(2)(A)$ & 2  \\
\end{tabular}
\end{center}

$G=\GL(3,\Field_2)$ has order 168 and 6 conjugacy classes:
\begin{center}
\begin{tabular}{r|l|c|c|c}
\# & representative(s) & char. polynomial & irrep & dim. \\
\hline
 1 & $\begin{bmatrix}1&.&.\\.&1&.\\.&.&1\end{bmatrix}$ & $(x+1)^3$  
    & $\yng(1,1,1)(A)$ & 1 \\
21 & $\begin{bmatrix}1&1&.\\.&1&.\\.&.&1\end{bmatrix}$ ,
     $\begin{bmatrix}1&.&.\\.&1&1\\.&.&1\end{bmatrix}$ & $(x+1)^3$  
    & $\yng(2,1)(A)$ , $\yng(1,2)(A)$ & 6  \\
24 & 
$\begin{bmatrix}1&1&.\\1&.&1\\.&1&.\end{bmatrix}$
,
$\begin{bmatrix}.&1&.\\1&.&1\\.&1&1\end{bmatrix}$
& $x^3+x^2+1$      & $C$ & 3 \\
24 & 
$\begin{bmatrix}1&1&.\\1&1&1\\.&1&.\end{bmatrix}$
,
$\begin{bmatrix}.&1&.\\1&1&1\\.&1&1\end{bmatrix}$
& $x^3+x+1$        & $C^{\star}$ & 3 \\
42 & $\begin{bmatrix}1&1&1\\.&1&1\\.&.&1\end{bmatrix}$
    ,  $\begin{bmatrix}1&1&.\\.&1&1\\.&.&1\end{bmatrix}$
    & $(x+1)^3$ 
    & $\yng(3)(A)$ & 8  \\
56 & $\begin{bmatrix}1&.&.\\.&1&1\\.&1&.\end{bmatrix}$ & $(x+1)(x^2+x+1)$
    & $A\tensor B$ & 7 \\
\hline
\strut =168 \\
\end{tabular}
\end{center}

%Here we show the dimension of the irreps:
%\renewcommand{\arraystretch}{1.4}
%\begin{center}
%\begin{tabular}{c|c}
%$d$  & irrep  \\
%\hline
%1    &  $\yng(1,1,1)(A)$ \\
%3    &  $C$ \\
%3    &  $C^{\star}$ \\
%6    &  $\yng(2,1)(A)$ \\
%7    &  $A\tensor B$ \\
%8    &  $\yng(3)(A)$ \\
%\end{tabular}
%\end{center}

$G=\GL(4,\Field_2)$ has order 20160 and 14 conjugacy classes:
\begin{center}
\begin{tabular}{r|l|c|c|c}
\# & representative & char. polynomial & irrep & dim. \\
\hline
1  & $\begin{bmatrix}1&.&.&.\\.&1&.&.\\.&.&1&.\\.&.&.&1\end{bmatrix}$  & $(x+1)^4$  & $\yng(1,1,1,1)(A)$ & 1  \\
105  & $\begin{bmatrix}1&1&.&.\\.&1&.&.\\.&.&1&.\\.&.&.&1\end{bmatrix}$  & $(x+1)^4$  & $\yng(2,1,1)(A)$ & ?  \\
112  & $\begin{bmatrix}1&1&.&.\\1&.&.&.\\.&.&1&1\\.&.&1&.\end{bmatrix}$  & $(x^2+x+1)^2$  & $\yng(1,1)(B)$ & ?  \\
210  & $\begin{bmatrix}1&1&.&.\\.&1&.&.\\.&.&1&1\\.&.&.&1\end{bmatrix}$  & $(x+1)^4$  & $\yng(2,2)(A)$ & ?  \\
1120  & $\begin{bmatrix}1&.&.&.\\.&1&.&.\\.&.&1&1\\.&.&1&.\end{bmatrix}$
  & $(x+1)^2(x^2+x+1)$  & $\yng(1,1)(A)\tensor B$ & ?  \\
1260  & $\begin{bmatrix}1&1&.&.\\.&1&1&.\\.&.&1&.\\.&.&.&1\end{bmatrix}$  & $(x+1)^4$  & $\yng(3,1)(A)$ & ?  \\
1344  & 
$\begin{bmatrix}.&1&.&.\\1&1&1&.\\.&1&.&1\\.&.&1&.\end{bmatrix}$,
$\begin{bmatrix}.&1&.&.\\1&.&1&.\\.&1&1&1\\.&.&1&.\end{bmatrix}$
& $x^4+x^3+x^2+x+1$  & $D$ & 21  \\
1344  & 
$\begin{bmatrix}1&1&.&.\\1&1&1&.\\.&1&.&1\\.&.&1&1\end{bmatrix}$,
$\begin{bmatrix}1&1&.&.\\1&.&1&.\\.&1&1&1\\.&.&1&1\end{bmatrix}$
& $x^4+x^3+1$        & $E$ & 21  \\
1344  & 
$\begin{bmatrix}.&1&.&.\\1&1&1&.\\.&1&.&1\\.&.&1&1\end{bmatrix}$,
$\begin{bmatrix}1&1&.&.\\1&.&1&.\\.&1&1&1\\.&.&1&.\end{bmatrix}$
& $x^4+x+1$          & $E^\star$ & 21  \\
1680  & $\begin{bmatrix}1&1&1&1\\1&.&1&.\\.&.&1&1\\.&.&1&.\end{bmatrix}$  & $(x^2+x+1)^2$  & $\yng(2)(B)$ & ?  \\
2520  & $\begin{bmatrix}1&1&.&.\\.&1&1&.\\.&.&1&1\\.&.&.&1\end{bmatrix}$  & $(x+1)^4$  & $\yng(4)(A)$ & 64   \\
2880  & $\begin{bmatrix}1&.&.&.\\.&1&1&.\\.&1&.&1\\.&.&1&.\end{bmatrix}$
  & $(x+1)(x^3+x^2+1)$  & $A\tensor C$ & 45   \\
2880  & $\begin{bmatrix}1&.&.&.\\.&1&1&.\\.&1&1&1\\.&.&1&.\end{bmatrix}$
  & $(x+1)(x^3+x+1)$  & $A\tensor C^{\star}$ & 45   \\
3360  & $\begin{bmatrix}1&1&.&.\\.&1&.&.\\.&.&1&1\\.&.&1&.\end{bmatrix}$  & $(x+1)^2(x^2+x+1)$  & $\yng(2)(A)\tensor B$ & ?  \\
\hline
\strut = 20160 \\
\end{tabular}
\end{center}

Where the irrep dimensions have square sum:
$$
 20160 = 1^2 + 7^2 + 14^2 + 20^2 + 21^2 + 21^2 + 21^2 + 28^2 + 35^2 + 45^2 + 45^2 + 56^2 + 64^2 + 70^2.
$$


\section{$\Sp(2n,\Field_p)$}

%See also:
%``Classification of the Subgroups of the Two-Qubit Clifford Group''
%    Eric Kubischta, Ian Teixeira
% https://arxiv.org/abs/2409.14624
% 

Note that $\Sp(2n,\Field_2) \cong \SO(2n+1,\Field_2)$.

$G=\Sp(4,\Field_2)$ has order 720 and 11 conjugacy classes:
\begin{center}
\begin{tabular}{r|l|c|c|c}
\# & representative & char. polynomial & irrep & dim. \\
\hline
1  & $\begin{bmatrix}1&.&.&.\\.&1&.&.\\.&.&1&.\\.&.&.&1\end{bmatrix}$  & $(x+1)^4$  & ? & 1  \\
15  & $\begin{bmatrix}1&.&.&.\\.&1&.&1\\1&.&1&.\\.&.&.&1\end{bmatrix}=CX_{0,1}$
    & $(x+1)^4$  & ? & ?  \\
15  & $\begin{bmatrix}1&1&.&.\\.&1&.&.\\.&.&1&.\\.&.&.&1\end{bmatrix}$  & $(x+1)^4$  & ? & ?  \\
40  &  $\begin{bmatrix}1&.&.&.\\.&1&.&.\\.&.&1&1\\.&.&1&.\end{bmatrix}$   & $(x+1)^2(x^2+x+1)$  & ? & ?  \\
40  & $\begin{bmatrix}1&1&.&.\\1&.&.&.\\.&.&1&1\\.&.&1&.\end{bmatrix}$    & $(x^2+x+1)^2$  & ? & ?  \\
45  & $\begin{bmatrix}1&1&.&.\\.&1&.&.\\.&.&1&1\\.&.&.&1\end{bmatrix}$  & $(x+1)^4$  & ? & ?  \\
90  &  $\begin{bmatrix}.&.&.&.\\.&.&.&.\\.&.&.&.\\.&.&.&.\end{bmatrix}$ ?  & $(x+1)^4$  & ? & ?  \\
90  & $\begin{bmatrix}.&.&.&.\\.&.&.&.\\.&.&.&.\\.&.&.&.\end{bmatrix}$  ?  & $(x+1)^4$  & ? & ?  \\
120  & $\begin{bmatrix}.&.&.&.\\.&.&.&.\\.&.&.&.\\.&.&.&.\end{bmatrix}$ ?   & $(x^2+x+1)^2$  & ? & ?  \\
120  & $\begin{bmatrix}1&1&.&.\\.&1&.&.\\.&.&1&1\\.&.&1&.\end{bmatrix}$     & $(x+1)^2(x^2+x+1)$  & ? & ?  \\
144  & $\begin{bmatrix}.&1&.&.\\1&1&1&.\\.&1&.&1\\.&.&1&.\end{bmatrix}$    & $x^4+x^3+x^2+x+1$  & $D$ & ?  \\
\hline
\strut = 720 \\
\end{tabular}
\end{center}


$G = \Sp(6,\Field_2)$  has order 1451520 and 30 conjugacy classes:

\begin{center}
\begin{tabular}{r|l|c|c|c}
\# & representative & char. polynomial & irrep & dim. \\
\hline
1  & & $(x+1)^6$  & ? & ?  \\
63  & & $(x+1)^6$  & ? & ?  \\
315  & & $(x+1)^6$  & ? & ?  \\
672  & & $(x+1)^4(x^2+x+1)$  & ? & ?  \\
945  & & $(x+1)^6$  & ? & ?  \\
2240  & & $(x^2+x+1)^3$  & ? & ?  \\
3780  & & $(x+1)^6$  & ? & ?  \\
3780  & & $(x+1)^6$  & ? & ?  \\
7560  & & $(x+1)^6$  & ? & ?  \\
7560  & & $(x+1)^6$  & ? & ?  \\
10080  & & $(x+1)^4(x^2+x+1)$  & ? & ?  \\
10080  & & $(x+1)^4(x^2+x+1)$  & ? & ?  \\
11340  & & $(x+1)^6$  & ? & ?  \\
13440  & & $(x+1)^2(x^2+x+1)^2$  & ? & ?  \\
20160  & & $(x^2+x+1)^3$  & ? & ?  \\
30240  & & $(x+1)^4(x^2+x+1)$  & ? & ?  \\
40320  & & $(x+1)^2(x^2+x+1)^2$  & ? & ?  \\
40320  & & $(x+1)^2(x^2+x+1)^2$  & ? & ?  \\
45360  & & $(x+1)^6$  & ? & ?  \\
48384  & & $(x+1)^2(x^4+x^3+x^2+x+1)$  & ? & ?  \\
60480  & & $(x+1)^4(x^2+x+1)$  & ? & ?  \\
60480  & & $(x+1)^4(x^2+x+1)$  & ? & ?  \\
90720  & & $(x+1)^6$  & ? & ?  \\
90720  & & $(x+1)^6$  & ? & ?  \\
96768  & & $(x^2+x+1)(x^4+x^3+x^2+x+1)$  & ? & ?  \\
120960  & & $(x^2+x+1)^3$  & ? & ?  \\
120960  & & $(x+1)^2(x^2+x+1)^2$  & ? & ?  \\
145152  & & $(x+1)^2(x^4+x^3+x^2+x+1)$  & ? & ?  \\
161280  & & $x^6+x^3+1$  & ? & ?  \\
207360  & & $(x^3+x+1)(x^3+x^2+1)$  & ? & ?  \\
\hline
\strut = 1451520 \\
\end{tabular}
\end{center}


\bibliography{refs2}{}
\bibliographystyle{abbrv}

\end{document}
